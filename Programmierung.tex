\section{Schwerpunkte der Programmierung} \label{Schwerpunkte der Programmierung}
Da aus Zeitgr\"unden nicht auf die gesamte Programmierung im einzelnen genau eingegangen werden kann, sind im folgenden die Schwerpunkte die sich bei der Programmierung ergaben genauer aufgef\"uhrt und werden anhand von Quellcode-Beispielen genauer beschrieben.

\subsection{Die Google-Places-API}
Die Google-Places-\ac{API} bietet eine Schnittstelle f\"ur die Anbindung an Google, mit welcher es m\"oglich ist verschiedene Orte zu suchen. Im Projekt wird die \ac{API} in der Activity Navigation verwendet, um einen Ort f\"ur die Navigation zu w\"ahlen (Bild \ref{Wireframe NaviAuswahl}).

In der Activity Navigation ist ein Textfeld zu finden, in welches der Nutzer einen Ort eintragen kann. Beim eintragen, wird der Nutzer \"uber eine Trefferliste bei der Auswahl unterst\"utz, wobei diese Vervollst\"andigung mit Hilfe der Places-\ac{API} geschieht.

Hierf\"ur wurde ein "`ArrayAdapter"' gew\"ahlt, welcher beim eintragen in das Textfeld eine Liste mit Treffern unterhalb des Feldes anzeigt.

Die statische Methode autocomplete() der Klasse GooglePlacesAutocompleteAdapter enth\"alt die Logik, mit der die Places-\ac{API} angesprochen wird.

Als erstes wird eine Ergebnisliste und eine HttpURLConnection deklariert, wobei durch einen StringBuilder eine URL gebaut und diese geladen wird.

Der StingBuilder baut einen Sting, \"ahnlich wie er im Listing zu sehen ist.
\lstinputlisting{Code/PlacesURL.txt}

Das daraufhin vom Google Server gelieferte JSON-Dokument wird dem StringBuilder hinzugef\"ugt, um es im n\"achsten Schritt weiter zu verarbeiten. Ein Beispiel f\"ur ein solches Dokument ist im Anhang \ref{Places_API-Response Anhang} zu finden.

Aus diesem JSON-Dokument wird nun ein Array mit den einzelnen M\"oglichkeiten erstellt, wobei aus diesem Array wiederum die einzelnen Beschreibungen unter dem Schl\"ussen "`description"` ausgelesen werdem. Diese Ergebnisse werden dann mit Hilfe der Ergebinsliste zur\"uckgegeben. Dieser Zusammenhang ist im Listing unten noch einmal als Quellcode dargestellt.
\cite{PlacesAPIGoogle}

\lstinputlisting{Code/Metode_autocomplete.java}
\cite{PlacesExample}

\subsection{ogg-Files und Android}
Im Lastenheft wurde festgelegt, dass die App zum einige Alarmt\"one selbst mitbringen soll, zum anderen sollen aber auch die internen Klingelt\"one verwendet werden k\"onnen.

Hierf\"ur wurden verschiedene Alarmt\"one zum Projekt hinzugef\"ugt, welche somit auch zu Auswahl stehen.

Bei dem Einbinden von standard Klingelt\"onen ergab sich jedoch ein schwerwiedendes Problem, welches im Folgenden genauer beschrieben wird.

Unter Android sind alle Klingelt\"one als sogenannte .ogg-Files abgelegt. Diese .ogg-Files k\"onnen ein Flag enthalten, welches Android anweist den Ton bis zum Abbruch der Wiedergabe ewig zu wiederholen. 
\cite{oogStackOver} (Abgerufen am 14.02.2015)

Hierbei wird auch der Aufruf der Mediaplayer-Methode setLooping(false) durch das ogg-Flag ignoriert. Hierbei handelt es sich um einen Dokumentierten Bug im Android-System.
\cite{oggBug} (Abgerufen am 14.02.2015)

Da der Alarmton beim anklicken durch den User demonstrativ abgespielt werden soll und dann automatisch enden soll, ergibt sich hierbei ein Problem, da sich Klingelt\"one unendlich wiederholen w\"urden.

Um dieses Problem zu um gehen, wurde ein Workaround geschrieben, welcher im Listing unten zu sehen ist.

Wird ein Ton gespielt, welcher ein Klingelton ist, so wird ein neuer Thread gestartet. Dieser Thread \"ubernimmt die Aufgabe die Wiedergabe zu unterbrechen, nachdem sie einmal abgespielt wurden ist.

Hierf\"ur wird eine While-Schleife gestartet, welche genau solange l\"auft, wie der eigentliche Ton, welcher gerade abgespielt wird. Ist diese Schleife beendet, wird die Wiedergabe unterbrochen und die Thread ordnungsgem\"a\ss{} beendet.

Das Listing ist im Quellcode in der Klasse SoundSelection unter der Methode createListAdapter() zu finden.

\lstinputlisting{Code/MediaplayerWorkaround.java}

\subsection{Observer- und Factory-Pattern}
Wie schon in den Unterkapitel \ref{AKD Factory Pattern} und \ref{AKD Observer Pattern} erw\"ahnt, musste f\"ur die Arbeit unter Android das Factory und das Observer Pattern angepasst werden. Im folgenden wird nun genauer auf die Programmierung eingegangen, da unter Android wie schon beschrieben nicht vorhergesagt werden kann, wo der Einsprungspunkt f\"ur eine App ist.

\subsubsection{Die Implementierung des Observer Pattern}
Da nicht bei jeder Verwendung des Observers angefragt werden soll, ob dieser schon existiert und auch eine \"Ubergabe der Observerinstanz mittels eines Android-Bundle unpraktikabel ist, wurde der Observer so entwurfen, dass er nur statische Methoden enth\"alt, welche von jeder Stelle des Quellcodes aus aufgerufen werden k\"onnen.

Zus\"atzlich h\"alt der Observer die beobachteten Instanzen nicht im Arbeitsspeicher, sondern legt diese mit Hilfe der Jackson-Library im Filesystem als JSON-Dokumente ab.
Dieses Vorgehen ist notwendig, da nicht vorhergesagt werden kann, wann eine App beendet und aus dem Arbeitsspeicher entfernt wird. Um den hieraus resultierenden Datenverlust zu verhindern, werden alle Instanzen im Filesystem abgelegt und k\"onnen \"uber den Observer von dort auch wieder gelesen werden.

Der Observer Mappt bei speichern, die Instanz zu einem JSON-Objekt, welches als File im Filesystem unter Android abgelegt wird. Hierbei gibt es unterschiedliche Methoden f\"ur das Speichern von SMS- und E-Mail-Regeln.

Das Lesen von Regeln geschieht in der umgekehrten Reihenfolge. Zuerst werden die Files, welche eine Regel beinhalten vom Filesystem gelesen. Anschliesend werden diese Files wieder zu Javainstanzen gemappt.

Die genaue Implementierung des Klasse RuleObserver ist im Anhang im Kapitel \ref{Die Klasse RuleObserver} zu finden.

Zus\"atzlich gibt es im Projekt auch noch die Klasse AlarmSettingsObserver, welche die Alarmeinstellungen beobachtet und die Klasse DepartmentObserver, welche die Ger\"atehauseinstellungen beobachtet. Beide Klassen sind \"ahnlich implentiert, wie die Klasse RuleObserver.

\subsubsection{Die Implementierung des Factory Pattern}
Aus den selben Gr\"unden wie beim Observer Pattern, musste auch das Factory Pattern unter Android angepasst werden.

Die Klasse RuleCreator stellt das Factory Pattern dar und hat verschiedene statische Methoden, die es erlauben eine Regelinstanz zu erstellen, oder zu \"andern.

\"Uber die Methode createRule() wird, unter der Angabe des Regelnames und des Typs, eine neue Regel der entsprechenden Instanz erstellt, und zur\"uckgeliefert.

Mit Hilfe der anderen Methoden kann eine Regel ge\"andert werden. So kann zum Beispiel der Absender oder der Alarmton ge\"andert werden. Hierbei folgen alle Methoden dem Schema wie es im Kapitel \ref{AKD Factory Pattern} schon einmal dargestellt wurde.
\lstinputlisting{Code/FactorySchema.java} 

Zus\"atzlich ruft jede Methode der Factory die Methode notifyObserver() der Klasse Rule auf, um den Observer \"uber eine Ver\"anderung zu informieren.

Die genaue Implementierung der Klasse RuleCreator ist im Anhang im Kapitel \ref{Die Klasse RuleCreator} zu finden. 

\subsection{Die Implementierung des E-Mail Receivers}
Da Android nicht wie vermutet einen eigenen Receiver f\"ur E-Mails bereitstellt, wie es bei den SMS der Fall ist, muss hierf\"ur eine eigene Implementierung erstellt werden.

Da diese Implementierung f\"ur alle Protokolle wie zum Beispiel IMAP oder SecureIMAP gemacht werden muss, wurde in der Arbeit aus Zeitgr\"unden nur die Implementierung des IMAP-Protokolls umgesetzt. Hierf\"ur wurde eine Bibliothek ben\"otigt, welche schon in der Lage ist Mails von einem Server zu lesen. Au\ss{}erdem musste darauf geachtet werden, dass diese auch mit Android kompatibel ist. Im Projekt wurde hierf\"ur die JavaMail-API f\"ur Android verwendet, welche Google bereitstellt. \cite{MailAPI}

F\"ur die Implementierung, wurde ein TimerTask verwendet, welcher in regelm\"a\ss{}igen Abst\"anden die Mails vom Server abruft. Dieser TimerTask ist in einem AsyncTask gekapselt, da unter Android ein Netzwerkzugriff nicht vom UI-Thread gestattet ist. 

Der AsyncTask wiederum wird von einem Service gestartet, welcher mit der Option \\"'START\_STICKY"` versehen ist. Dies bedeutet, dass der Service vom System immer neu gestartet wird, sollte er beendet werden.

Nach einigen Tests dieser Prototypischen Implementierung wurden verschiedene Fehlverhalten festgestellt wie zum Beispiel, dass der Service nach einer gewissen Zeit beendet wird.
Einge genauere Untersuchung und Behebung der Fehler beim abrufen der Mails konnte aus zeitlichen Gr\"unden nicht fertig gestellt werden.

Da der Mail Receiver nicht stabil lauff\"ahig ist, wurde er nicht im Projekt selbst implentiert. Der Prototyp ist aber auf der der Arbeit beiliegenden CD zu finden.

\subsection{Versenden einer Regel}
LF60 des Lastenhefts verlangt, das Regeln \"ubertragen werden k\"onnen sollen. Dies bedeutet, dass es einem Nutzer m\"oglich sein muss eine seiner Regeln auch an andere Nutzer der App versenden zu k\"onnen. Hierbei gibt das Lastenheft keine konkreten Vorgaben, sondern nur einige Vorschl\"age wie eine Regel \"ubertragen werden kann.

Der Nutzer hat in der Activity der Regel\"ubersicht die m\"oglichkeit eine Regel l\"anger zu ber\"uhren. \"Uber das daraufhin erscheinende Contextmen\"u ist es m\"oglich eine Regel zu \"ubertragen.

Damit der Nutzer viele verschiedene M\"oglichkeiten hat eine Regel zu \"ubertragen, wurde ein Intent gew\"ahlt um den Nutzer eine Auswahl der Sendem\"oglichkeiten zur Verf\"ugung zu stellen.

Es wird also ein impliziter Intent erzeugt, welcher es erlaubt dem Nutzer auszuw\"ahlen welchen Versandweg er nutzen m\"ochte. Dem Intent wird ein JSON-Dokument \"ubergeben, welches mit dem Mimetype "'text/plain"` versehen ist.

Damit andere Activitys die Datei auch lesen k\"onnen, muss die Regel vor dem Start der vom Nutzer gew\"ahlten Activity zuerst an einen \"offentlichen Speicherplatz kopiert werden.
Dies ist notwendig, da andere Apps nicht auf den internen Speicherbereich anderer Apps zugreifen k\"onnen.

Der eben beschriebene Zusammenhang ist im folgenden Quellcodeausschnitt noch einmal dargestellt. \cite{AttachFileToIntent}

\lstinputlisting{Code/SendRuleIntent.java}

Zus\"atzlich ergibt sich durch den gew\"ahlten Weg auch die M\"oglichkeit, dass der Nutzer seine Regeln zum Beispiel in Google Drive oder Dropbox speichern kann. Dies ist davon abh\"angig, welche Apps der Nutzer installiert hat.