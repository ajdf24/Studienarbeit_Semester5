\section{Schwerpunkte der Programmierung}
Da aus Zeitgr\"unden nicht auf die gesamte Programmierung im einzelnen genau eingegangen werden kann, sind im Folgenden Schwerpunkte die sich bei der Programmierung ergaben genauer aufgef\"uhrt und werden anhand von Quellcode-Beispielen genauer beschrieben.

\subsection{Die Google-Places-API}
Die Google-Places-\ac{API} bietet eine Schnittstelle f\"ur die Anbindung an Google, mit welcher es m\"oglich ist verschiedene Orte zu suchen. Im Projekt wird die \ac{API} in der Activity Navigation verwendet, um einen Ort f\"ur die Navigation zu w\"ahlen (Bild \ref{Wireframe NaviAuswahl}).

Im Projekt ist in der Activity Navigation ein Textfeld, in welches der Nutzer einen Ort eintragen kann. Beim eintragen, wird der Nutzer \"uber eine Trefferliste bei der Auswahl unterst\"utz, wobei diese Vervollst\"andigung mit Hilfe der Places-\ac{API} geschieht.

Hierf\"ur wurde ein "`ArrayAdapter"' gew\"ahlt welcher beim eintragen in das Textfeld eine Liste mit Treffern unterhalb des Textfeldes darstellt.

Die statische Methode autocomplete() der Klasse GooglePlacesAutocompleteAdapter enth\"alt die Logik, mit der die Places-\ac{API} angesprochen wird.

Als erstes wird eine Ergebnisliste und eine HttpURLConnection deklariert, wobei durch einen StringBuilder eine URL gebaut und geladen wird.

Der StingBuilder baut einen Sting, wie er im Listing zu sehen ist.
\lstinputlisting{Code/PlacesURL.txt}

Das daraufhin zum Google Server gelieferte JSON-Dokument wird dem StringBuilder hinzugef\"ugt um es im n\"achsten Schritt weiter zu verarbeiten. Ein Beispiel f\"ur ein solches ist im Anhang \ref{Places_API-Response Anhang} zu finden.

Aus diesem JSON-Dokument wird nun ein Array mit den einzelnen M\"oglichkeiten erstellt, wobei aus diesem Array wiederum die einzelnen Beschreibungen "description"` ausgelesen werdem. Diese Ergebnisse werden dann mit Hilfe der Ergebinsliste zur\"uckgegeben. Dieser Zusammenhang ist im Listing unten noch einmal als Quellcode dargestellt.
\cite{PlacesAPIGoogle}

\lstinputlisting{Code/Metode_autocomplete.java}
\cite{PlacesExample}

\subsection{ogg-Files und Android}
\cite{oggBug} \cite{oogStackOver}

\subsection{Observer- und Factory-Pattern}