\section{Bevor es losgehen kann - PC und Smartphone einrichten}
Bevor es mit der eigentlichen Programmierarbeit losgehen kann, muss zuerst das schon erw\"ahnte Android \ac{SDK}, von der Google-Developer-Seite\footnote{\url{http://developer.android.com/sdk/index.html}} heruntergeladen und installiert werden. Ist dies erledigt, kann es eigentlich schon fast losgehen.

\subsection{Der Android SDK Manager} \label{Der Android SDK Manager}
Bevor die Programmierarbeit losgehen kann muss nach der installation zuerst der "`Android SDK Manager"' gestartet werden. Dies ist zum einen wichtig, um die neusten Updates zu laden, aber auch um die passenden APIs zu laden.

Wie schon erw�hnt, ist der Manager zum einen die Update-Plattform des \ac{SDK}-Packets und zum anderen auch der Installer f�r neue Packete.

Beim starten des \ac{SDK} Managers werden Updates f�r schon installierte Packete angeboten. Damit der Download und die Installation startet, muss der Nutzer diesen Vorgang best�tigen. Sind alle Updates geladen und werden keine neuen Packete ben�tigt kann der Manager wieder geschlossen werden.

Beim ersten Starten des Managers muss aber noch ausgew�hlt werden auf welchem API-Level (welcher Android-Version) programmiert werden soll. Da Android relativ weit abw�rtskompatibel ist, empfiehlt es sich bei einem neuen Projekt immer die neuste API zu installieren. 

Durch die Verwendung der neusten API ist zum einen garantiert, dass das Projekt auf der neusten Android-Version l\"auft, zum anderen ist sichergestellt, dass das Projekt relativ lange ohne \"Anderungen mit zuk\"unftigen Versionen lauff\"ahig bleibt.