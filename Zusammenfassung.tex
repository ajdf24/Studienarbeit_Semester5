\section{Zusammenfassung}
Die vorliegende Studienarbeit befasst sich mit der Entwicklung einer Android Applikation, welche empfangene SMS und E-Mails auswertet und den Nutzer \"uber den Eingang einer solchen Nachricht informiert. Dies ist notwendig, um Einsatzkr\"afte der Freiwilligen Feuerwehr zum Beispiel \"uber den Erhalt einer AlarmSMS gesondert zu informieren.

Die App wurde nach einem Lastenheft modelliert, welches im Kapitel \ref{Lastenheft} zu finden ist. Im Zuge dieser Arbeit wurde das mobile Betriebssystem Android, welches von Google Inc. stammt genauer analysiert. Bei der Analyse wurde das System im allgemeinen betrachtet. Zus\"atzlich wurde auch analysiert, wie eine App aus Nutzer- beziehungsweise aus Programmierer sicht aussieht. 

Es wurde darauf eingegangen, wie ein PC und ein Smartphone f\"ur die Entwicklung von Apps vorbereitet werden muss. Da zum Beginn der Arbeit die \ac{IDE} "`Android Studio"' sich noch in der Betaphase befand, wurde sich im Kapitel \ref{Bevor es losgehen kann} nur auf die Arbeit mit der \ac{IDE} "`Eclipse"' bezogen. 

Nach einer grundlegenden Analyse des Androidsystems und dessen Apps, wurde das Lastenheft analysiert und dargestellt, welche Punkte sich wie ums\"atzen lassen. Hierbei wurde sich immer wieder auf die vorherigen Kapitel bezogen.

Durch die Analyse des Lastenhefts, wurden viele Erkenntnisse gesammelt, welche in mehreren Use-Case-Diagrammen untergebracht wurden sind. So entstand w\"ahrend der Arbeit ein Diagramm, mit zus\"atzlichen f\"unf Verfeinerungen, um alle Use-Cases genau zu beschreiben.

Aus den erstellten Use-Case-Diagrammen und der Lastenheftanalyse wurden wiederum Wireframes erstellt, welche im Kapitel \ref{Wireframes} zu finden sind. Die Wireframes bilden alle m\"oglichen Bildschirme der App und deren Zusammenhang ab. Eine komplette \"Ubersicht aller Wireframes ist auf der beiliegenden CD zu finden, da ein volls\"andiger Ausdruck gr\"o\ss{}enbedingt nicht m\"oglich war.

Aus den Wireframes, den Use-Cases und dem Lastenheft wurde anschlie\ss{}end ein Analyseklassendiagramm gefertigt, welches einen \"Uberblick \"uber den Quellcode und dessen Aufbau geben soll. Zus\"atzlich wurden im Kapitel \ref{Das AKD} auch die im Klassendiagramm vorhandenen Programmierpattern und das Model-View-Controller-Prinzip erl\"autert, nach denen die App programmiert wurde.

Anschlie\ss{}end wurde der Unterschied zwischen dem Facebook \ac{SDK} und dem Post \"uber einen Intent erl\"autert, da die App eigentlich automatisch auf Facebook Posten sollte. Dies konnte jedoch durch Beschr\"ankungen in der Facebook Policy IV.2. nicht umgesetzt werden. Stattdessen wurde ein Post auf dem Sozialen Netzwerk Twitter umgesetzt. N\"aheres hierzu ist im Kapitel \ref{Facebook SDK} zu finden.

Da nicht im einzelnen auf die ganze Programmierung der App eingegangen werden konnte, wurden im Kapitel \ref{Schwerpunkte der Programmierung} die Schwerpunkte der Programmierung genauer dargestellt. Der geammte Quellcode ist jedoch auf der beiliegenden CD zu finden.

Android verf\"ugt \"uber keinen E-Mail-Receiver, wie es bei den SMS der Fall ist. Daher musste eine Art Receiver selbst Implementiert werden. Im Verlauf der Arbeit wurde hierf\"ur ein Prototyp implementiert, welcher jedoch aus Zeitgr\"unden nicht so stabiel Lauff\"ahig gemacht werden konnte, um ihn in das Projekt zu integrieren. N\"aheres hierzu ist ebenfalls im Kapitel \ref{Schwerpunkte der Programmierung} zu finden.

Zus\"atzlich erschien im Verlauf der Arbeit die neuen Androidversionen 5.0 und 5.1, welche den Codenamen Lollipop tragen. Die App wurde im laufe der Arbeit zwar auf die neuen Versionen migriert, eine genauere Beschreibung der beiden Versionen und speziell die Beschreibung der neuen Designsprache "`Material"' konnten aus Zeitgr\"unden jedoch nicht umgesetzt werden.

