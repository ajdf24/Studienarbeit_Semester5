\section{\"Ubertragung des Lastenhefts in Wireframes}
\label{Wireframes}

Bei der Bearbeitung des Projekts wurde entschieden, nach der Lastenheftanalyse und der Erstellung der Use Cases, zuerst die gesamte GUI nach der Aufgabenstellung des Lastenhefts zu modellieden. Durch diese Vorgehensweise wird schneller ersichtlich, welche Android-Komponenten genau ben\"otigt werden und wie sie zu implementieren sind.

F\"ur die Modellierung der Wireframes wurde das Mockup-Tool WireframeSketcher verwendet, welches \"ahnlich wie \ac{ADT} auch ein Eclipse-Plugin ist. 

Nach einer ersten Erstellung der Wireframes wird eine Evaluierung mit Hilfe der Kameraden der Freiwilligen Feuerwehr Rastenberg stattfinden, bei der eventuelle Schwachstellen oder Verbesserungen diskutiert werden sollen.

\subsection{Die Wireframes}

Im Bild \ref{Wireframe StartBildschirm} sind die ersten zwei Wireframes zu sehen. Im linken Frame ist der Start-Bildschirm der App abgebildet, welcher beim Starten der App angezeigt werden soll. Hier hat der Nutzer eine \"Ubersicht \"uber alle seine Regeln, welche mit Hilfe der Checkboxen rechts aktiviert oder deaktiviert werden k\"onnen, wie es das Lastenheft nach LF70 verlangt. In der oberen Men\"uleisten ist zum einen ein Plus-Symbol, sowie zum anderen das standardm\"a\ss{}ige Android-Symbol f\"ur das Activity abh\"angige Popupmen\"u zu sehen.

Tippt der Nutzer nun auf das Plus-Symbol, gelangt er zu einer neuen Activity (im Bild \ref{Wireframe StartBildschirm} rechts), mit der er eine neue Regel erzeugen kann.
Die neue Activity verlangt zum einen einen Namen f\"ur die Regel, zum anderen eine Angabe, ob es sich bei der Regel um eine SMS-Regel oder eine E-Mail-Regel handelt. Die Auswahl der Regelart, ist \"uber die Radiobuttons unter dem Textfeld m\"oglich.

\begin{figure}[!ht]
\centering
\includegraphics[width=10cm]{Bilder/StartBildschirm.png}
\caption{Wireframe des StartBildschirms mit der Erstellung einer Regel}
\label{Wireframe StartBildschirm}
\centering
\end{figure}

Im unteren Teil der Activity hat der Nutzer zwei Buttons, zum einen, um die Aktion abzubrechen und zum anderen einen Button, um weitere Regeleinstellungen vorzunehmen.
Beim Bet\"atigen des Weiter-Buttons gelangt der Nutzer auf eine weitere Activity, welche im Bild \ref{Wireframe Regeluebersicht} dargestellt ist.

\begin{figure}[!ht]
\centering
\includegraphics[width=16cm]{Bilder/WireframeRegeluebersicht.png}
\caption{Wireframe der Regel\"ubersicht und der Absenderauswahl}
\label{Wireframe Regeluebersicht}
\centering
\end{figure}

Das Bild \ref{Wireframe Regeluebersicht} zeigt die n\"achsten Wireframes. Im linken Frame ist die \"Ubersicht \"uber alle Einstellungsm\"oglichkeiten einer Regel zu finden. Tippt der Nutzer nun auf eine dieser Auswahlm\"oglichkeiten, so gelangt er zur jeweiligen Activity. Im weiteren Verlauf werden alle aufgef\"uhrten Activitys dargestellt und genauer erl\"autert.
Alle nun folgenden Activitys sehen gleich aus, egal ob es sich nun um eine SMS-Regel oder E-Mail-Regel handelt.

Wurde die Absenderauswahl gew\"ahlt, so gelangt man zu einer Activity, welche es erm\"oglicht, einen Absender einzugeben. Hier hat der Nutzer die Wahl, ob er selbst eine Nummer hinzuf\"ugt oder ob er \"uber den Button "`Auswahl aus Kontakten"' einen Kontakt w\"ahlt. Wird nun eben dieser Button get\"atigt, so gelangt der Nutzer zu einem Content Provider (im Bild \ref{Wireframe Regeluebersicht} rechts dargestellt), welcher alle Kontakte auflistet. 

Hier ist es m\"oglich, einen Kontakt \"uber die Suchleiste zu finden, oder einfach nach unten zu scrollen. Wird ein Kontakt ausgew\"ahlt, so gelangt der Nutzer automatisiert zur Absenderauswahl zur\"uck und der gew\"ahlte Kontakt wird ins Absender-Feld eingetragen.

Je nachdem, ob es sich um eine SMS- oder E-Mail-Regel handelt, wird die Telefonnummer beziehungsweise die E-Mail-Adresse eingetragen.

Ist die Auswahl vollendet, gelangt der Nutzer \"uber den Speichern-Button zur\"uck zur linken Activity, wo er weitere Einstellungen vornehmen kann.
 
\begin{figure}[!ht]
\centering
\includegraphics[width=10cm]{Bilder/WireframeWortwahl.png}
\caption{Wireframe der Wortauswahl und der Alarmtonwahl}
\label{Wireframe Wortauswahl}
\centering
\end{figure}

Tippt der Nutzer nun auf die Wortauswahl (im Bild \ref{Wireframe Regeluebersicht} links), so gelangt er auf eine Activity, welche es ihm erm\"oglicht, Schl\"usselw\"orter auszuw\"ahlen. Im Bild \ref{Wireframe Wortauswahl} ist die Activity mit der Schlagwortauswahl links dargestellt. 

Der Nutzer hat nun die M\"oglichkeit, Schlagw\"orter zu w\"ahlen, welche vorkommen m\"ussen und welche nicht enthalten sein d\"urfen. Sind beide Felder leer, so reicht es aus, das eine Nachricht vom Absender eingeht, um einen Alarm auszul\"osen.

Sollten die Felder einen Inhalt haben, so wird die Nachricht auf diese geparst. Sollten die Informationen zutreffen, das hei\ss{}t, sind alle W\"orter vorhanden, beziehungsweise sind ausgeschlossene W\"orter nicht vorhanden, wird ein Alarm abgesetzt. 

Die einzelnen W\"orter m\"ussen, wie in der Activity beschrieben, durch ein Leerzeichen getrennt werden.

In der rechten Activity ist die Tonauswahl dargestellt, hier hat der Nutzer die M\"oglichkeit, einen Alarmton auszuw\"ahlen, welcher beim Zutreffen der Regel abgespielt wird. In der Auswahl sind die Standard-T\"one des Smartphones und die mit der App mitgelieferten T\"one zu finden.

Im Bild \ref{Wireframe Antwort} ist in der linken Activity die Erstellung der automatisierten Antwort dargestellt. Wie schon im Bild \ref{Wireframe Regeluebersicht} ist die Empf\"angerauswahl wieder manuell und \"uber einen Content Provider geregelt. Im Textfeld darunter kann der Nutzer eine Nachricht eingeben, welche automatisch gesendet wird, sollte beim Empfangen einer Alarmnachricht eine gewisse Entferung zum Ger\"atehaus \"uberschritten sein.

Die Entfernung kann der Nutzer nat\"urlich auch ausw\"ahlen und zwar \"uber eine Listenauswahl, welche im unteren Teil der Activity zu finden ist. Um alle informationen zu speichern, gibt es auch in dieser Activity wieder einen Speichern-Button.

In der rechten Activity im Bild \ref{Wireframe Antwort} ist die Voreinstellung f\"ur einen Facebookpost zu treffen. Der Nutzer kann einen Standardtext angeben, der beim Erhalt einer Nachricht automatisch auf Facebook gepostet wird. Auch ist es m\"oglich, \"uber die Checkbox den Nachrichtentext mit an den Post anzuh\"angen. Um einen Facebookpost zu ver\"offentlichen, muss entweder die App der Facebook AG auf dem Ger\"at installiert sein oder die Facebook API in das Projekt eingebaut werden. Welcher Weg der praktikablere ist, wird im Kapitel \ref{Facebook SDK} genauer erl\"autert.
\begin{figure}[!ht]
\centering
\includegraphics[width=16cm]{Bilder/WireframeAntwort.png}
\caption{Wireframe der automatisierten Antwort und des Facebookposts}
\label{Wireframe Antwort}
\centering
\end{figure}

Im Bild \ref{Wireframe NaviAuswahl} ist die Activity zum Einstellen eines Navigationsziels zu sehen. Hier besteht zum einen die M\"oglickeit, eine Adresse selbstst\"andig einzugeben oder mit automatischer Google-Erg\"anzung, was mit Hilfe der Places-\ac{API} von Google geschieht. Ist eine Adresse gew\"ahlt, muss der Nutzer diese, wie schon \"ofter erl\"autert, mit dem Speichern-Button sichern.

Hiermit wurden nun alle Activitys und M\"oglichkeiten dargestellt, die eine Regelerstellung bietet.

\begin{figure}[!ht]
\centering
\includegraphics[width=5cm]{Bilder/WireframeNaviAuswahl.png}
\caption{Wireframe der Navigationsauswahl}
\label{Wireframe NaviAuswahl}
\centering
\end{figure}
\FloatBarrier

Im Bild \ref{Wireframe Regeluebersicht Popup} links ist wieder die Regelauswahl zu sehen, diesmal jedoch mit ge\"offnetem Popup-Men\"u. Das Men\"u enth\"alt die drei Punkte Neue Regel, Alarmeinstellungen und Ger\"atehauseinstellung. Der erste Men\"upunkt f\"uhrt wieder zu der Regelerstellung, wie oben schon beschrieben. 

Der zweite Punkt \"offnet das globale Einstellungsmen\"u f\"ur den Alarmempfang, dies ist im Wireframe rechts dargestellt. Als erstes ist ein Schalter f\"ur die Aktivierung der Alarmierung zu finden. Ist dieser deaktiviert, so werden keine Alarme angezeigt. Als n\"achstes kann der Nutzer die Lautst\"arke des Alarms \"uber den Schieberegler ausw\"ahlen. Dieser regelt die Lautst\"arke, mit welcher ein Alarm abgeben wird.

Als n\"achstes folgt der Schalter, an dem eingestellt werden kann, ob ein Alarm mit Vibration oder ohne signalisiert werden soll. Darauf folgt wieder ein Schalter, der das Notification Light bei einer Alarmierung aktiviert. Hierf\"ur gibt es nat\"urlich direkt darunter eine Auswahl der Lichtfarbe, mit der alarmiert werden soll.

Der letzte Schalter ist der wahrscheinlich wichtigste, denn er gibt an, ob eine Alarmierung auch erfolgen soll, wenn das Telefon sich im Lautlosmodus befindet.

\begin{figure}[!ht]
\centering
\includegraphics[width=10cm]{Bilder/WireframeRegeluebersicht_popup.png}
\caption{Wireframe der Regel\"ubersicht mit Popup und Alarmeinstellungen}
\label{Wireframe Regeluebersicht Popup}
\centering
\end{figure}
\FloatBarrier
Die letzte noch fehlende Activity ist im Wireframe im Bild \ref{Wireframe Geraetehaus} zu sehen. In diese Activity gelangt man \"uber den dritten Popupmen\"upunkt. Die Activity enth\"alt einzig ein Textfeld, in das der Standort des Ger\"atehauses mit Hilfe der Google-Autovervollst\"andigung (Places-\ac{API}) eingetragen werden kann.

Mit diesem Standort errechnet die App sp\"ater die Entferung zum Ger\"atehaus.
\begin{figure}[!ht]
\centering
\includegraphics[width=5cm]{Bilder/WireframeGeraetehaus.png}
\caption{Wireframe der Ger\"atehauseinstellung}
\label{Wireframe Geraetehaus}
\centering
\end{figure}
\FloatBarrier
% \newpage

\subsection{Die Evaluierung}
Die eben gezeigten und genauer beschriebenen Wireframes wurden zusammen mit einer kurzen Pr\"asentation des Projekts der Freiwilligen Feuerwehr Rastenberg vorgestellt.
Hierbei wurde eine Evaluierung des Projekts vorgenommen. Zusammen mit den Kameraden wurde der Einsatzbereich der App noch einmal genau analysiert. Hierbei lag der Schwerpunkt auf einer Einsatzalarmierung per SMS. 

Nachdem die Kameraden in die Grundfunktionen eingewiesen waren, fanden sie das Konzept schl\"ussig und gut abgerundet. Sie hatten keinerlei \"Anderungsvorschl\"age bei bestehenden Funktionen. 

Jedoch kam der Wunsch auf, die empfangene SMS zus\"atzlich zur Alarmierung auch vorzulesen, da im Alarmfall keine Zeit besteht, auf das Smartphone zu schauen. Gleicherma\ss{}en sollten weitere, also Benachrichtigungen, die andere als automatisierte Antworten schicken, ebenfalls vorgelesen werden.

Hierbei wurde betont, dass man mit den vorgelesenen Informationen, bereits auf dem Weg zum Ger\"atehaus, eine Planung und Einteilung der Einsatzkr\"afte vornehmen k\"onnte.

Als Folge daraus wurden die bestehenden Wireframes zum Teil angepasst.

\subsection{Die Wireframes nach der Evaluierung}
Nach der Evaluation des Projekts wurden die Wireframes noch einmal \"uberarbeitet, um den neuen Anforderungen gerecht zu werden.

Im Bild \ref{Wireframe Regeluebersicht nach Eval} ist wie im Bild \ref{Wireframe Regeluebersicht} die Liste mit allen Einstellungsm\"oglichkeiten einer Regel zu finden. Neu ist der letzte Punkt "`Automatisches Vorlesen"'. Tippt der Nutzer auf diese Einstellung, gelangt er zu einer neuen Activity, welche im Bild \ref{Wireframe Vorlesen} genauer gezeigt ist.
\begin{figure}[!ht]
\centering
\includegraphics[width=5cm]{Bilder/WireFrame_RegelauswahlNachEvaluation.png}
\caption{Wireframe der Regel\"ubersicht nach der Evaluation}
\label{Wireframe Regeluebersicht nach Eval}
\centering
\end{figure}

Im folgenden Bild \ref{Wireframe Vorlesen} ist, wie eben schon erl\"autert, die Activity zur Einstellung der automatischen Vorlesung von Nachrichtigen zu sehen.
Der Nutzer hat die Wahl, die eingehende Nachricht, welche die Regel matcht, vorlesen zu lassen, und alle nachfolgenden. Hierbei werden alle eingehenden Nachrichten vorgelesen, welche innerhalb eines Zeitfensters von f\"unf Minuten eintreffen.
\begin{figure}[!ht]
\centering
\includegraphics[width=5cm]{Bilder/WireFrameVorlesen.png}
\caption{Wireframe der Einstellung zum automatischen Vorlesen}
\label{Wireframe Vorlesen}
\centering
\end{figure}
\FloatBarrier

\subsection{Auswirkungen der Evaluierung auf das Lastenheft}
Nach der Evaluierung muss das Lastenheft somit um einen weiteren Punkt erweitert werden.

\begin{minipage}{3cm}
/LF110/
\end{minipage}
\begin{minipage}{12,2cm}
Die App soll beim Erhalt einer Alarmierung nicht nur einen Ton spielen, sondern auch den Inhalt der AlarmSMS vorlesen k\"onnen. Zus\"atzlich sollen alle nachfolgenden Nachrichten innerhalb von f\"unf Minuten auch vorgelesen werden, um die Planung von Einsatzkr\"aften schon auf dem Weg zum Ger\"atehaus beginnen zu k\"onnen.\\
\end{minipage}