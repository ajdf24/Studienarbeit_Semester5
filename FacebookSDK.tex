\section{Facebook SDK vs. Intent} \label{Facebook SDK}
Um einen Post auf Facebook zu erm\"oglichen, gibt es verschiedene M\"oglichkeiten. Zum einen besteht die M\"oglichkeit diesen Post mit Hilfe der Facebook App und einem Intent zu l\"o\ss{}en. Zum anderen gibt es auch das Facebook SDK, welches es erm\"oglicht einen Post direkt an Facebook zu senden, ohne andere Apps mit einbeziehen zu m\"ussen.

\subsection{Post \"uber einen Intent}
Generell besteht immer die M\"oglichkeit, unterschiedliche Apps \"uber einen Intent kommunizieren zu lassen. Hierf\"ur muss eine App, einen Intent kreieren und eine andere muss diesen Intent auswerten und verarbeiten k\"onnen.

Bei der Facebook App ist es nach aktuellem Stand nicht m\"oglich eine Plain-Text-Nachricht \"uber einen Intent zu \"ubergeben. Lediglich ein Link, ein Bild und so weiter k\"onnen \"ubergeben werden. Leider ist es durch die Facebook Policy IV.2. nicht mehr m\"oglich einen Post f\"ur den Nutzer vorab zu f\"ullen. \cite{fbPolicy2}


\subsection{Post \"uber das Facebook SDK}
\"Uber das Facebook SDK ist es m\"oglich, die App direkt bei Facebook zu registrieren und dienste von Facebook in Anspruch zu nehmen, wie zum Beispiel das abrufen der Freunde.
Hierf\"ur, muss sich der Nutzer in der App bei Facebook anmelden, um sich zu authentifizieren. Ist dies geschehen, kann die App die vorher bei Facebook angegebenen Rechte nutzen.

Leider ist es auch mit dem Facebook SDK nicht m\"oglich, einen Post f\"ur den Nutzer automatisch mit Plain-Text zu bef\"ullen. Auch bei dem SDK ist nat\"urlich die Facebook Policy IV.2. zu beachten, welche es nicht erlaubt einen Post mit Plain-Text zu bef\"ullen. \cite{fbPolicy2} 

\subsection{Auswirkungen auf das Projekt}
Aus den eben genannten Gr\"unden, ist es leider nicht m\"oglich einen Facebook Post wie im Lastenheft erw\"unscht, und in den Wireframes beschrieben umzus\"atzen.
M\"oglich w\"are jedoch eine unterst\"utzung f\"ur andere soziale Netzwerke wie zum Beispiel Twitter.

Im Projekt wurde ein Twitter-Post \"uber einen einfachen Intent realisiert. Hierf\"ur muss jedoch die entsprechende App von Twitter Inc. auf dem Smartphone installiert sein. Ist dies nicht der Fall, so kann auch kein Post auf Twitter erstellt werden.