\section{Aufgabenstellung}
Mit Hilfe einer mobilen Android-Applikation sollen Feuerwehreinsatzkr�fte durch eine eingehende "`Alarm-SMS"' benachrichtigt werden, welche von Leitstellen selbst oder von Privatfirmen stammt, welche die Funkinformationen der Leitstellen auswerten. Die gesendeten SMS sind allerdings nicht als Alarmmeldung sofort identifizierbar, sondern kommen wie jede andere SMS beim Empf�nger an.
Beim Erhalt einer solchen SMS soll der Nutzer des Telefons akustisch �ber einstellbare Klingelt�ne sowie visuell �ber das in Android-Ger�ten �bliche "`Notification-Light"' informiert werden.

Eine Alarm-SMS besteht, wie jede andere SMS auch, zum einen aus dem Absendernamen, welcher eine Telefonnummer oder eine Zeichenkette sein kann. Zum anderen enth�lt die SMS einen Nachrichtentext, in welchem bestimmte Informationen, wie Name der Feuerwehr, Einsatzzeit und Einsatzstichwort stehen.

Um eine Unabh�ngigkeit vom Absendesystem zu erreichen, soll der Nutzer zuerst eine beliebige Zeichenfolge eingeben, auf welche die SMS gepr�ft wird. Hierf�r muss ein Dienst entwickelt werden, welcher jede eingehende SMS pr�ft.

Sollte es sich um eine Alarm-SMS handeln, so soll der Nutzer umgehend darauf besonders hingewiesen werden. Hierf�r soll �berpr�ft werden, welche M�glichkeiten f�r eine besondere Alarmierung im Android-System bestehen.

Der Nutzer soll �ber ein neutrales Regelbasiertes System neue Regeln definieren, diese bearbeiten und l�schen k�nnen. 
Hierf\"ur muss ein System ausgearbeitet werden, welches funktional und nutzerfreundlich ist. Auch soll der Export beziehungsweise der Import von Regeln m\"oglich sein, um diese untereinander austauschen zu k\"onnen.

% Bsp: Absender-Nummer.equals("+4972160825769") AND SMS.getContent().contains("foo")
% ==> Action.Klingelton("Mambo5")
Eine weiteres wichtiges Ziel des zu entwickelnden Systems ist, dass es sich bei einem Neustart des Smartphones automatisch mit dem System startet und somit umgehend, ohne Nutzer Aktionen, einsatzbereit ist.
F�r diesen Zweck m\"ussen die eingegebenen Regeln gespeichert werden. Hierf�r bietet sich zum einen die Speicherung in einer Datei an, aber auch die Speicherung in einer Datenbank soll in Betracht gezogen werden, da mit ihrer Hilfe auch eine Statistik �ber Eins�tze gemacht werden kann.
Zus�tzlich sollen verschiedene Alarmt�ne zur Auswahl stehen, unter denen der Nutzer w�hlen kann.

Zus\"atzlich zur eigentlichen Alarmierungsfunktion soll untersucht werden, wie sich eine Post-Funktion f\"ur "`Social Networks"' wie Twitter, Facebook oder Google+ umsetzen l\"asst. Sollte sich ein String mit dem Einsatzort in der SMS befinden, so soll die Applikation die M\"oglichkeit bieten eine Navigation zu starten.

Zum Abschluss soll untersucht werden, wie es m\"oglich ist eine SMS oder beziehungsweise eine E-Mail zu senden, um den Einheitsf\"uhrer \"uber ein m\"ogliches kommen beziehungsweise ein Fernbleiben zu melden.

