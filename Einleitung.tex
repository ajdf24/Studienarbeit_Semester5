\section{Einleitung}
Es gibt heute verschiedene M�glichkeiten, die Einsatzkr�fte der Freiwilligen Feuerwehren im Einsatzfall zu alarmieren, wobei die beste und schnellste M�glichkeit einen Alarm abzusetzen, die Verwendung von Funkmeldeempf�ngern ist. Jedoch haben vor allem kleine St�dte und Gemeinden nicht die finanziellen M�glichkeiten, jede Einsatzkraft mit einem dieser Ger�te auszustatten, da nicht nur ihre Anschaffung, sondern auch die Wartung und Reparatur sehr teuer ist. 

Aus diesem Grund wird, vor allem im l�ndlichen Bereich, noch oft auf eine Luftsirene gesetzt, da hier alle Kr�fte mit einem einzigen nicht wartungsaufw�ndigen Ger�t alarmiert werden k�nnen. Doch auch hier ergeben sich viele Nachteile. So sind zum einen Beschallungspl�ne st�ndig fort�zuschreiben, was haupts�chlich durch Neubauten geschuldet ist. Gegebenenfalls muss dann der Standort der Sirene ver�ndert oder weitere angeschafft werden, um eine komplette Abdeckung zu gew�hrleisten. Weiterhin machen neue Baubestimmungen eine Alarmierung zusehends schwerer, da zum Beispiel doppelt oder dreifach verglaste Fenster die Au�enger�usche extrem d�mpfen.

Eine weitere M�glichkeit den Einsatzkr�ften Bescheid zu geben, ist die der  SMS-Alarmierung. Im Einsatzfall wird jedem Mitglied der Feuerwehr eine SMS gesendet. Diese Variante ist sehr Kosteng�nstig f�r St�dte und Gemeinden, da nicht spezielle Empf�nger angeschafft werden m�ssen. Denn in der heutigen Zeit hat praktisch jeder ein Smartphone in seinem Besitz. Nachteilig hier ist, dass nicht jeder Nutzer sein Ger�t im h�rbaren Modus betreibt und somit einen Alarm eventuell nicht mitbekommt. 

Da die SMS von einen Server gesendet wird, gibt es keine Absender-Nummer, sondern nur einen String als Absender. Hieraus ergeben sich weitere Nachteile. Zum einen kann kein spezieller Ton hinzuf�gen werden (Nummerabh�nger Ton), da die ankommende SMS keine Absender-Nummer besitzt und zum anderen ist es bei nahezu allen Ger�ten nicht m�glich, den Lautlos-Modus bei bestimmten Nummern auszusetzen und trotz der stillen Benachrichtigung einen Ton zu spielen. Hierf\"ur muss Abhilfe geschaffen werden, denn eine Alarm-SMS ist von besonderer Wichtigkeit und sollte auf jeden Fall wahrgenommen werden.

Mit Hilfe einer mobilen Applikation k�nnten die genannten Nachteile einer freien SMS-Alarmierung aufgehoben werden. Auf dem freien Markt gibt es schon einige Applikationen, die diese Aufgabe erf�llen, doch sind diese immer an spezielle Benachrichtigungssysteme gebunden, welche teuer lizenziert werden.

F\"ur die Einsatzkr\"afte der Feuerwehr, sowie f\"ur St\"adte und Gemeinden, w\"are es Sinnvoll eine Lizenzkostenfreie Applikation zur Verf\"ugung zu haben. Dies w\"urde nicht nur Kosten sparen, sondern im Ernstfall sogar Leben retten, da die Einsatzkr\"afte zuverl\"assig \"uber einen Einsatz informiert werden und zu Hilfe eilen k\"onnen.

Da das mobile Betriebssystem Android mit circa 84,6 Prozent Marktanteil am h\"aufigsten anzutreffen ist, soll die Applikation auf diesem Betriebssystem aufsetzen. Eine Adaption der Applikation auf "`iOS"' und "`Windows Phone"' ist zwar angedacht, wird aber im Zuge dieser Arbeit nicht umgesetzt. \cite{GolemMobileBetriebssystem}
